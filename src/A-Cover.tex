\thispagestyle{empty}   % 定义起始页的页眉页脚格式为 empty —— 空,也就没有页眉页脚

\begin{table}[]
    \centering
    \renewcommand\arraystretch{2}
    \begin{tabular}{|l|l|}
    \hline
    \quad队伍编号\quad & MC2311870\quad \\ \hline
    \quad题号   & B         \\ \hline
    \end{tabular}
\end{table}



\begin{center}
    \textbf{—————————————————————————————————}

    \textbf{\fontsize{20}{1.5}总城市轨道交通列车时刻表优化问题的研究}

    \textbf{摘 要}
\end{center}





% ==================================================
%
%   摘要
%
% --------------------------------------------------

接下来是摘要。\cite{2023NianDiShiSanJieMathorCupBiSaiSaiKeGuanWang}接下来还是摘要。接下来时摘要。接下来还是摘要。接下来时摘要。接下来还是摘要。接下来时摘要。接下来还是摘要。接下来时摘要。接下来还是摘要。接下来时摘要。接下来还是摘要。接下来时摘要。接下来还是摘要。接下来时摘要。接下来还是摘要。接下来时摘要。接下来还是摘要。接下来时摘要。接下来还是摘要。接下来时摘要。接下来还是摘要。接下来时摘要。接下来还是摘要。接下来时摘要。接下来还是摘要。接下来时摘要。接下来还是摘要。接下来时摘要。接下来还是摘要。接下来时摘要。接下来还是摘要。接下来时摘要。接下来还是摘要。接下来时摘要。接下来还是摘要。接下来时摘要。接下来还是摘要。接下来时摘要。接下来还是摘要。接下来时摘要。接下来还是摘要。接下来时摘要。接下来还是摘要。接下来时摘要。接下来还是摘要。接下来时摘要。接下来还是摘要。接下来时摘要。接下来还是摘要。接下来时摘要。接下来还是摘要。接下来时摘要。接下来还是摘要。接下来时摘要。接下来还是摘要。接下来时摘要。接下来还是摘要。接下来时摘要。接下来还是摘要。接下来时摘要。接下来还是摘要。接下来时摘要。接下来还是摘要。接下来时摘要。接下来还是摘要。接下来时摘要。接下来还是摘要。接下来时摘要。接下来还是摘要。接下来时摘要。接下来还是摘要。接下来时摘要。接下来还是摘要。接下来时摘要。接下来还是摘要。接下来时摘要。接下来还是摘要。接下来时摘要。接下来还是摘要。

\textbf{关键词}:关键词一 关键词二 关键词三




% ==================================================
%
%   目录
%
% --------------------------------------------------

% 下面添加一个目录,同样定义目录页眉页脚格式为 empty —— 空
\newpage
\tableofcontents
\thispagestyle{empty}