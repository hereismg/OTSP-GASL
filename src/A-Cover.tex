\thispagestyle{empty}   % 定义起始页的页眉页脚格式为 empty —— 空,也就没有页眉页脚

\begin{table}[]
    \centering
    \renewcommand\arraystretch{2}
    \begin{tabular}{|l|l|}
    \hline
    \quad队伍编号\quad & MC2311870\quad \\ \hline
    \quad题号   & B         \\ \hline
    \end{tabular}
\end{table}



\begin{center}
    \textbf{—————————————————————————————————}

    \textbf{\fontsize{20}{1.5}通过遗传算法和仿真模拟优化列车时刻表问题}

    \textbf{摘 要}
\end{center}





% ==================================================
%
%   摘要
%
% --------------------------------------------------

列车时刻表优化问题一直以来都是交通运输以及运筹管理学界重点研究的问题,一般而言,该问题也是强耦合、多目标的NP类问题。大多数情况下,需要借助计算机强大的算力辅助研究,通过实现相关的搜索算法,找到一种较为合理的解决方案\cite{niuGuidaoliecheshikebiaowentiyanjiuzongshu2021}。

在解决题设问题之前,

\textbf{关键词}:多目标优化 遗传算法 仿真模拟 动态规划 列车时刻表优化问题




% ==================================================
%
%   目录
%
% --------------------------------------------------

% 下面添加一个目录,同样定义目录页眉页脚格式为 empty —— 空
\newpage
\tableofcontents
\thispagestyle{empty}