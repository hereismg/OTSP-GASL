\thispagestyle{empty}   % 定义起始页的页眉页脚格式为 empty —— 空,也就没有页眉页脚

% \begin{table}[h]
%     \centering
%     \renewcommand\arraystretch{2}
%     \begin{tabular}{|l|l|}
%     \hline
%     \quad队伍编号\quad & MC2311870\quad \\ \hline
%     \quad题号   & B         \\ \hline
%     \end{tabular}
% \end{table}

\begin{table}[h]
    \centering
    \begin{tabular}{|c|c|}
    \hline
    队伍编号 & MC2311870 \\ \hline
    题号   & B         \\ \hline
    \end{tabular}
\end{table}



\begin{center}
    \textbf{—————————————————————————————————}

    \textbf{\fontsize{20}{1.5}通过遗传算法和仿真模拟优化列车时刻表问题}

    \textbf{摘 要}
\end{center}





% ==================================================
%
%   摘要
%
% --------------------------------------------------

列车时刻表优化问题一直以来都是交通运输以及运筹管理学界重点研究的问题,一般而言,该问题也是强耦合、多目标的NP类问题。大多数情况下,需要借助计算机强大的计算能力来辅助研究,通过实现相关的搜索算法,找到一种较为合理的解决方案\cite{niuGuidaoliecheshikebiaowentiyanjiuzongshu2021}。

为了更好地建立模型优化列车时刻表问题,本文首先建立了用于描述列车运行状态的\textbf{列车客运模型},并且初步阐述了\textbf{遗传算法}的基本原理以及遗传算法框图。对于列车客运模型,本文设计了一套仿真算法,模拟列车运行过程,以规划列车的运行过程,避免列车之间的冲突。规划完成以后,可以计算出该列车交路方案的运营成本与服务质量。

做好模型准备之后,开始建立模型解决题设给出的三个问题。

\textbf{针对问题一},本文主要通过Python语言,设计计算机模拟仿真算法,计算列车客运模型,绘制列车运行图。在规划列车运行的过程中,考虑各种条件限制,把列车的行程安排在一个合理的区间之中。最后,通过遗传算法,搜索出较为可行的方案。

\textbf{针对问题二},本文通过仿真模拟算法,计算列车客运模型,绘制列车运行图,然后通过运行Python代码输出列车的时刻表以及等间隔的平行运行线路图。


\textbf{针对问题三},本文通过分析疫情开放前后的客流量数据,提出修改旧列车开行方案的必要性,并且针对工作日、日常节假日和春节三种客流量不同的情况,提出对应的\textbf{动态调整方案}。\newline
\newline
\textbf{关键词}:多目标优化 遗传算法 仿真模拟 列车时刻表优化问题



% ==================================================
%
%   目录
%
% --------------------------------------------------

% 下面添加一个目录,同样定义目录页眉页脚格式为 empty —— 空
\newpage
\tableofcontents
\thispagestyle{empty}