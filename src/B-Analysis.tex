\setcounter{page}{1}        % 将页码计数器设置为 1

% ==================================================
%
%   问题重述
%
% --------------------------------------------------

\section{问题重述}

列车时刻表问题是一类NP难问题\cite{caoJiyuchengkedengdaishijiandechengshiguidaojiaotongliecheshikebiaoyouhuamoxingyusuanfayanjiu2021}。列车时刻表规定了每趟列车的起点和终点,让每个零散的列车构成一个有机的整体,更好的服务于人民群众。

为了更好的给出列车组的问题,要设计出列车开行方案,再设计列车开行方案中,主要解决下面三个小问题:

\begin{enumerate}
    \item 列车编组方案,在本文中,每趟列车的车厢型号相同,列车编组数量一致;
    \item 列车停站方案,在本文中,每趟列车每个经过的车站都会停车;
    \item 列车交路计划,在本文中,考虑使用\textbf{大小交路方案}。
\end{enumerate}

% ==================================================
%
%
%
% --------------------------------------------------

\section{问题分析}

% ==================================================
%
%
%
% --------------------------------------------------

\section{模型假设与符号说明}

\subsection{模型假设}

\begin{enumerate}
    \item 假设每趟列车的车厢型号相同,列车编组数量一致;
    \item 假设每趟列车每个站点都会停车;
    \item 假设用户的数量是一定的。
\end{enumerate}

\subsection{符号说明}

\begin{table}[h]
    \centering
    \begin{tabular}{@{}cc@{}}
    \toprule
    符号         & 意义                              \\ \midrule
    $M_f$       & OD客流矩阵                          \\
    $M_f(i,j)$  & 从车站$i$到车站$j$的乘客数量(人)            \\
    $t_i$       & 列车从车站$i$行驶到车站$i+1$所需要的时间($s$) \\
    $t_{avg}$   & 乘客平均上下车时间($s$/人)                  \\
    $P_{in}(i)$         &  在车站$i$上车的人数(人)    \\
    $P_{out}(i)$        & 在车站$i$下车的人数(人)   \\
    $w_i$       & 在车站$i$上下车的总耗时($s$) \\
    $t_{sum}$   & 所有乘客总等待时间($s$)     \\
                &                                 \\
                &                                 \\
                &                                 \\
                &                                 \\
                &                                 \\
                &                                 \\ \bottomrule
    \end{tabular}
\end{table}
