\setcounter{page}{1}        % 将页码计数器设置为 1

% ==================================================
%
%   问题重述
%
% --------------------------------------------------

\section{问题重述}



列车时刻表问题是一类NP难问题\cite{caoJiyuchengkedengdaishijiandechengshiguidaojiaotongliecheshikebiaoyouhuamoxingyusuanfayanjiu2021}。列车时刻表规定了每趟列车的起点和终点,使每趟零散的列车有机地组成一个整体,为乘客提供更好的服务。

为了更好的给出列车组的问题,要设计出列车开行方案,再设计列车开行方案中,主要解决下面三个小问题:

\begin{enumerate}
    \item 列车编组方案,在本文中,每趟列车的车厢型号相同,列车编组数量一致;
    \item 列车停站方案,在本文中,每趟列车每个经过的车站都会停车;
    \item 列车交路计划,在本文中,考虑使用\textbf{大小交路方案}。
\end{enumerate}

以降低运营成本并提高服务质量为目标,设计列车开行方案,并给出列车的等间隔平行运行图。

在解决整个问题的过程中,主要考虑发车间隔时间的设置,也为保证列车行驶的安全性,每个列车的追踪间隔不能过短。

% ==================================================
%
%   问题分析
%
% --------------------------------------------------

\section{问题分析}

原题的难点在于如何描绘出列车运行情况,以及如何解决列车行驶的冲突问题,也要满足不同列车之间的最小追踪间隔。综合以上考虑,所以本文主要采用仿真模拟来确定列车的运行状态。最后通过遗传算法求解最优解。

\subsection{对于问题一}

首先建立最基本的\textbf{列车客运模型},用来描述整个列车系统的状态,绘制出列车运行图帮助理解列车模式。建立好列车客运模型之后,通过\textbf{仿真模拟算法}模拟整个系统的运动状态,计算出时刻表,此时,可以计算出该方案的运营成本和服务质量。

由于每个如果列车交并方案不同,运营成本和服务质量也会不相同。所以经过多次的计算模拟,求出所有方案的运营成本和服务质量,以目标函数为\textbf{帕累托最优解集}求出最终答案。

\subsection{对于问题二}

根据原题目意思所述,以运营成本最小化,服务质量最大化为目标,设计列车时刻表。同样通过\textbf{仿真模拟}得出解决方案,最后以表格的形式给出列车的时刻表。

\subsection{对于问题三}

在现实中,不同时间短的旅客数量是决然不同的,所以要想进一步优化运营成本和服务质量,应该根据不同的时间段配置不同可列车开行方案。在本文中,主要考虑了工作日列车开行方案、日常节假日的列车开行方案、春节假期的列车开行方案。



% ==================================================
%
%   模型假设与符号说
%
% --------------------------------------------------

\section{模型假设与符号说明}

\subsection{模型假设}

为了方便考虑问题,本文在不影响准确性的前提下,做出如下几个假设:

\begin{enumerate}
    \item 假设每趟列车的车厢型号相同,列车编组数量一致;
    \item 假设每趟列车每个站点都会停车;
    \item 假设每个乘客都是文明乘客,均遵守文明道德,让车上的乘客先下车,自己再上车。
    \item 假设所有乘客都会再7:00准时来到车站等车。
\end{enumerate}

\subsection{符号说明}

下文给出本文主要的数学符号,其他细节符号会在文中出现时做出具体的说明。

\begin{table}[h]
    \begin{tabular}{
    >{\columncolor[HTML]{FEFEFE}}c 
    >{\columncolor[HTML]{FEFEFE}}c }
    \hline
    {\color[HTML]{333333} \textbf{符号}}         & {\color[HTML]{333333} \textbf{意义}}                                                 \\ \hline
    {\color[HTML]{333333} $\alpha(i,j)$}       & {\color[HTML]{333333} 车站$i$到车站b的乘客比例}                                                \\
    {\color[HTML]{333333} $M_f(i, j)$}         & {\color[HTML]{333333} 站台$i$中打算去往站台$j$的乘客人数}                                            \\
    {\color[HTML]{333333} $M_{in}(i, j, t)$}   & {\color[HTML]{333333} 站台$i$乘客打算从第t辆发车的列车到达站台$j$的人数}                                    \\
    {\color[HTML]{333333} $M_{cur}(i, j, t)$} & {\color[HTML]{333333} 第t辆发车的列车刚到车站$i$时, 其内打算去往站台$j$的乘客人数}                              \\
    {\color[HTML]{333333} $M_{out}(i, t)$}    & {\color[HTML]{333333} 第t辆发车的列车到达车站$i$的乘客下车人数}                                        \\
    {\color[HTML]{333333} $N_l : N_s$}         & {\color[HTML]{333333} 大交与小交的滚动发车比($N_l = 1 \textbackslash{}\text{or} N_s = 1$)} \\
    {\color[HTML]{333333} $s$}                   & {\color[HTML]{333333} 小交起点站台编号}                                                    \\
    {\color[HTML]{333333} $e$}                   & {\color[HTML]{333333} 小交终点站台编号}                                                    \\
    {\color[HTML]{333333} $V$}                   & {\color[HTML]{333333} 列车定员}                                                        \\
    {\color[HTML]{333333} $t_{max}$}          & {\color[HTML]{333333} 列车的最大等待时间}                                                   \\
    {\color[HTML]{333333} $M_{tm}(i, t)$}     & {\color[HTML]{333333} 第t辆出发的列车刚从车站i出发距离第一辆车开行所用时间}                                 \\
    {\color[HTML]{333333} $t_{total}$}        & {\color[HTML]{333333} 完成所有载客需要的最短时间}                                               \\
    {\color[HTML]{333333} $t_{del}$}          & {\color[HTML]{333333} 乘客平均上下车时间}                                                   \\
    {\color[HTML]{333333} $t_i$}                & {\color[HTML]{333333} 从站台i到站台$i+1$的运行时间}                                             \\
    {\color[HTML]{333333} $t_{dis}$}          & {\color[HTML]{333333} 最小追踪间隔时间}                                                    \\ \hline
    \end{tabular}
\end{table}