\setcounter{page}{1}        % 将页码计数器设置为 1

% ==================================================
%
%   问题重述
%
% --------------------------------------------------

\section{问题重述}



列车时刻表问题是一类NP难问题\cite{caoJiyuchengkedengdaishijiandechengshiguidaojiaotongliecheshikebiaoyouhuamoxingyusuanfayanjiu2021}。列车时刻表规定了每趟列车的起点和终点,让每个零散的列车构成一个有机的整体,为群众提供更好的服务。

为了更好的给出列车组的问题,要设计出列车开行方案,再设计列车开行方案中,主要解决下面三个小问题:

\begin{enumerate}
    \item 列车编组方案,在本文中,每趟列车的车厢型号相同,列车编组数量一致;
    \item 列车停站方案,在本文中,每趟列车每个经过的车站都会停车;
    \item 列车交路计划,在本文中,考虑使用\textbf{大小交路方案}。
\end{enumerate}

以降低运营成本和提高服务质量为目标,设计列车开行方案,并给出列车的等间隔平行运行图。

在解决整个问题的过程中,主要考虑发车间隔时间的设置,也为了保证列车行驶的安全性,每个列车的追踪间隔不能太短。

% ==================================================
%
%   问题分析
%
% --------------------------------------------------

\section{问题分析}

原题的难点在于如何描绘出列车运行情况,以及如何解决列车行驶的冲突问题,也要满足不同列车之间的最小追踪间隔。综合以上考虑,所以本文主要采用仿真模拟来确定列车的运行状态。最后通过遗传算法求解最优解。

\subsection{对于问题一}

首先建立最基本的\textbf{列车客运模型},用来描述整个列车系统的状态,绘制出列车运行图帮助理解列车模式。建立好列车客运模型之后,通过\textbf{仿真模拟算法}模拟整个系统的运动状态,计算出时刻表,此时,可以计算出该方案的运营成本和服务质量。

由于每个如果列车交并方案不同,运营成本和服务质量也会不相同。所以经过多次的计算模拟,求出所有方案的运营成本和服务质量,以目标函数为\textbf{帕累托最优解集}求出最终答案。

\subsection{对于问题二}

根据原题目意思所述,以运营成本最小化,服务质量最大化为目标,设计列车时刻表。同样通过\textbf{仿真模拟}得出解决方案,最后以表格的形式给出列车的时刻表。

\subsection{对于问题三}



% ==================================================
%
%   模型假设与符号说
%
% --------------------------------------------------

\section{模型假设与符号说明}

\subsection{模型假设}

为了方便考虑问题,本文在不影响准确性的前提下,做出如下几个假设:

\begin{enumerate}
    \item 假设每趟列车的车厢型号相同,列车编组数量一致;
    \item 假设每趟列车每个站点都会停车;
    \item 假设每个乘客都是文明乘客,均遵守文明道德,让车上的乘客先下车,自己再上车。
    \item 假设所有乘客都会再7:00准时来到车站等车。
\end{enumerate}

\subsection{符号说明}

下文给出本文主要的数学模型,如果有任何问题,请给出

\begin{table}[h]
    \centering
    \begin{tabular}{@{}cc@{}}
    \toprule
    符号         & 意义                              \\ \midrule
    $M_f$       & OD客流矩阵                          \\
    $M_f(i,j)$  & 从车站$i$到车站$j$的乘客数量(人)            \\
    $t_i$       & 列车从车站$i$行驶到车站$i+1$所需要的时间($s$) \\
    $t_{avg}$   & 乘客平均上下车时间($s$/人)                  \\
    $P_{in}(i)$         &  在车站$i$上车的人数(人)    \\
    $P_{out}(i)$        & 在车站$i$下车的人数(人)   \\
    $w_i$       & 在车站$i$上下车的总耗时($s$) \\
    $t_{sum}$   & 所有乘客总等待时间($s$)     \\
                &                                 \\
    $M_{in}$    & 乘客上车矩阵           \\
    $M_{in}(i,j)$    & 列车在站台$i$时乘客在站台$j$下车的实际上车人数(人)     \\
    $V_{out}$    & 乘客下车向量           \\
    $M_{out}(i)$    & 乘客在站台$i$的实际下车人数(人)     \\
    $M_{cur}$    & 列车内乘客矩阵           \\
    $M_{cur}(i,j)$    & 列车刚到站台$i$时, 其内乘客到站台$j$的人数(人)     \\
                &                                 \\
                &                                 \\
                &                                 \\ \bottomrule
    \end{tabular}
\end{table}
